\subsection{Billedekompression}

Datakompression refererer til den process hvori man reducerer mængden af data der er nødvendig for at vise en mængde information. 

Kompressionsratioen er beskrevet ved Ligning~\ref{comp-ratio}, hvor $b$ er det originale data og $b'$ er den komprimerede.

\begin{equation}\label{comp-ratio}
C = \frac{b}{b'}
\end{equation}

Den relative \textit{data redundancy} kan derved beskrives ved Ligning~\ref{eq:data-redundandy}, hvor R beskriver hvor stor en del af den originale mængde data, som var redundant.

\begin{equation}\label{eq:data-redundandy}
R = 1-\frac{1}{C}
\end{equation}

\subsubsection{Typer af Data-redundans}

\begin{enumerate}
	\item \textbf{Coding redundancy}\\
	Information skrives som bits. Eksempelvis kan noget information beskrives med mindre en 8-bit.
	
	\item \textbf{Spatial og temporary redundancy}\\
	Pixels er ofte ens eller afhængige af sine naboer og kan derfor ofte genskabes af færre bits.
	
	\item \textbf{Irrelevant information}\\
	Eksempelvis visuel information som ignoreres af menneskets øje.
\end{enumerate}

\subsubsection{Lossless vs. Lossy Kompression}

\begin{itemize}
	\item \textbf{Lossless}\\
	Originale billede kan genskabens. 
	
	\item \textbf{Lossy}\\
	Originale billede kan \textit{ikke} genskabes.
\end{itemize}
